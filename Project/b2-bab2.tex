%==================================================================
% Ini adalah bab 2
% Silahkan edit sesuai kebutuhan, baik menambah atau mengurangi \section, \subsection
%==================================================================

\chapter[LANDASAN TEORI DAN KERANGKA PEMIKIRAN]{\\ LANDASAN TEORI DAN KERANGKA PEMIKIRAN}

Bab ini merupakan penjelasan tentang landasan teori yang digunakan dalam penelitian dan kerangka pemikiran meliputi:

\section{Tinjauan Pustaka}
\begin{sectioncontent}
    \hspace{\parindent}Merupakan suatu penjelasan tentang hasil penelitian lain yang pernah dilakukan oleh  peneliti lain yang ada kaitan dengan penelitian yang akan dilakukan. Bagian ini juga menjelaskan masalah-masalah yang belum terpecahkan atau belum terjawab oleh penelitian terdahulu. Secara umum, bagian Tinjauan Pustaka berfungsi menjelaskan posisi penelitian yang dilakukan penulis di antara penelitian-penelitian terdahulu. Untuk dapat menjelaskan posisi ini, penulis harus memahami penelitian-penelitian yang telah dilakukan peneliti lain, lengkap dengan konteks yang melatar belakanginya, termasuk kritik atau komentar terhadap hasil dan temuan dari penelitan tersebut. Ketajaman dalam melakukan penelaahan dan kritik serta pengetahuan tentang peta perkembangan penelitian yang relevan akan membuka kemudahan peneliti dalam menyusun kerangka pemikiran pemecahan masalah, perumusan hipotesis (jika ada), dan pemilihan metode penelitian yang akan digunakan. Minimal hasil penelitian / jurnal berjumlah 5 dan diterbitkan 5 tahun terakhir sejak penulisan karya ilmiah dilakukan.
\end{sectioncontent}

\section{Landasan Teori}
\begin{sectioncontent}
    \hspace{\parindent}Merupakan suatu penjelasan tentang sumber acuan terbaru dari pustaka primer seperti artikel, jurnal, monograf, dan tulisan asli lainnya untuk mengetahui perkembangan penelitian yang relevan dengan judul atau tema penelitian yang akan dilakukan dan juga sebagai arahan dalam memecahkan masalah yang diteliti. Dalam hal ini, landasan teori dapat berupa suatu uraian yang bersifat kualitatif, suatu model matematis, ataupun bentuk-bentuk representatif yang lain. Kutipan, cuplikan, dan saduran dari literatur ditulis dengan menyebutkan penulis dan tahun sumber pustaka yang diacunya.
\end{sectioncontent}

% Implementasi untuk bagian Kerangka Pemikiran.

\newpage % Untuk memisahkan halaman Kerangka Pemikiran ke dalam sebuah halaman baru (tersendiri). Nonaktifkan (beri tanda % di depan) untuk menggunakan halaman yang sama dengan section sebelumnya.

% Gunakan opsi ini untuk menampilkan gambar yang dirancang untuk tampilan landscape.
\thispagestyle{empty}% Hilangkan nomor halaman default
\atxy{\dimexpr\paperwidth-0.5in}{0.5\paperheight}{\rotatebox[origin=center]{90}{\thepage}}
\begin{sidewaysfigure}

\section{Kerangka Pemikiran}
\begin{sectioncontent}
    \begin{figure}[H]
        \centering
        \includegraphics[scale=1]{logo-ubs.png} % Silakan sesuaikan ukuran dan gambar yang relevan
        \caption{Kerangka Pemikiran}
        \label{fig:logo-ubs} % Label untuk referensi gambar
    \end{figure}
\end{sectioncontent}

\end{sidewaysfigure}
% Selesai landscape - halaman selanjutnya akan kembali ke portrait

% Gunakan opsi ini untuk menampilkan gambar yang dirancang untuk tampilan portrait
% \section{Kerangka Pemikiran}
% \begin{sectioncontent}
%     % \hspace{\parindent}Merupakan suatu penjelasan tentang kerangka berpikir untuk memecahkan masalah yang sedang diteliti, termasuk menguraikan objek penelitian. Untuk melengkapi uraian kerangka pemikiran, peneliti dapat menyajikan kerangka pemikiran dalam bentuk diagram.
%     \begin{figure}[H]
%         \centering
%         \includegraphics[scale=1]{logo-ubs.png} % Silakan sesuaikan ukuran dan gambar yang relevan
%         \caption{Kerangka Pemikiran}
%         \label{fig:logo-ubs} % Label untuk referensi gambar
%     \end{figure}
% \end{sectioncontent}