%==================================================================
% Ini adalah kata pengantar
%==================================================================

%% DILARANG EDIT BAGIAN INI
\clearpage
\phantomsection
\addcontentsline{toc}{chapter}{KATA PENGANTAR}
\begin{center}
    \textbf{\large KATA PENGANTAR}\\[3em]
\end{center}
%% DILARANG EDIT BAGIAN INI

%% Edit bagian ini sesuai kebutuhan
Dengan memanjatkan puji syukur kehadiran Allah SWT yang telah melimpahkan segala rahmat dan hidayahnya kepada Penulis, sehingga tersusunlah Skripsi yang berjudul "\textbf{\judulid}"

Skripsi ini merupakan salah satu persyaratan yang diajukan dalam rangka menempuh ujian akhir untuk memperoleh gelar {\gelar} {\gelarsingkat} pada Program Studi {\prodi}, Program Studi {\prodi} di {\fakultas} {\universitas}.

Penulis sungguh sangat menyadari, bahwa penulisan Skripsi ini tidak akan terwujud tanpa adanya dukungan dan bantuan dari berbagai pihak terutama Ayahanda dan Ibunda serta yang lainnya. Maka, dalam kesempatan ini penulis menghaturkan penghargaan dan ucapan terima kasih yang sebesar-besarnya kepada :

\begin{enumerate}[leftmargin=0.5cm]
    \item Ibu \textbf{\rektor}, selaku Rektor Universitas Bani Saleh, yang telah memberikan kesempatan belajar bagi penulis untuk dapat menyelesaikan program Sarjana di kampus tercinta ini.
    \item Bapak \textbf{\dekan}, selaku Dekan Fakultas Teknologi Informasi dan Digital Universitas Bani Saleh, yang telah memberikan kesempatan belajar bagi penulis untuk dapat menyelesaikan program Sarjana di kampus tercinta ini.
    \item Bapak/Ibu \textbf{\kaprodi}, sebagai Ka Prodi Fakultas Teknologi Informasi dan Digital Universitas Bani Saleh, yang banyak membantu penulis dalam mengarahkan penulisan skripsi.
    \item Bapak/Ibu \textbf{\pembimbingutama}, sebagai Pembimbing utama Skripsi dan dosen yang dengan sabar dan tekun memberikan arahan perbaikan yang berarti bagi penulis.
    \item Bapak/Ibu \textbf{\pembimbingpendamping}, sebagai Pembimbing pendamping Skripsi dan dosen yang telah membagi ilmu pengetahuan dan pengalaman serta membimbing materi skripi ini.
    \item Rekan kuliah yang selalu memberikan motivasi, teman diskusi dalam hal penyelesaian skripsi dan banyak memberikan pencerahan.
    \item Rekan-rekan se-angkatan yang telah saling memberikan bantuan dan dukungan moral agar dapat terselesainya skripsi ini.
    \item Semua pihak yang tidak bisa disebutkan satu per satu yang telah memberikan dukungan dan bantuan atas segala hal yang terkait dengan terselesaikannya skripsi.
\end{enumerate}

Akhir kata, dengan keterbatasan yang ada pada penulis tentunya masih banyak kekurangan dan masih jauh dari kesempurnaan, hanya Allah SWT yang memiliki segala kesempurnaan. Oleh sebab itu masukan berupa kritik dan saran yang membangun akan sangat membantu bagi penulis. Semoga skripsi ini dapat memberikan manfaat bagi khasanah pengetahuan Teknologi Informasi di Indonesia.
%% Edit bagian ini sesuai kebutuhan

%% DILARANG EDIT BAGIAN INI
% \begin{flushright}
%     {\kota}, \tglpengesahan\\[1.25cm]
%     \penulis \\
%     \nim
% \end{flushright}
%% DILARANG EDIT BAGIAN INI