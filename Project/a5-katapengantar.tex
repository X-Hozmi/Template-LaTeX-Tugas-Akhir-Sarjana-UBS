%==================================================================
% Ini adalah kata pengantar
%==================================================================

%% DILARANG EDIT BAGIAN INI
\clearpage
\phantomsection
\addcontentsline{toc}{chapter}{KATA PENGANTAR}
\begin{center}
    \textbf{\large KATA PENGANTAR}\\[3em]
\end{center}
%% DILARANG EDIT BAGIAN INI

%% Edit bagian ini sesuai kebutuhan
Dengan memanjatkan puji syukur kehadiran Allah SWT yang telah melimpahkan segala rahmat dan hidayahnya kepada Penulis, sehingga tersusunlah Skripsi yang berjudul "\textbf{\judulid}"

Skripsi ini merupakan salah satu persyaratan yang diajukan dalam rangka menempuh ujian akhir untuk memperoleh gelar {\gelar} {\gelarsingkat} pada Program Studi {\prodi}, Program Studi {\prodi} di {\fakultas} {\universitas}.

Penulis sungguh sangat menyadari, bahwa penulisan Skripsi ini tidak akan terwujud tanpa adanya dukungan dan bantuan dari berbagai pihak terutama Ayahanda dan Ibunda serta yang lainnya. Maka, dalam kesempatan ini penulis menghaturkan penghargaan dan ucapan terima kasih yang sebesar-besarnya kepada :

\begin{enumerate}
    \item {\pembimbingutama} selaku Dosen Pembimbing TA yang telah banyak memberikan semangat, dorongan, dan bimbingan selama penyusunan Tugas Akhir ini.
    \item {\pembimbingutama}, {\sekretaris}, {\penguji} selaku Ketua Penguji, Sekretaris, dan Penguji yang sudah  memberikan koreksi perbaikan secara komprehensif terhadap TA ini.
    \item {\koorprodi} selaku Ketua Program Studi Sarjana Terapan Teknik Elektronika beserta dosen dan staf yang telah memberikan bantuan dan fasilitas selama proses penyusunan pra proposal sampai dengan selesainya TA ini.
    \item Semua pihak, secara langsung maupun tidak langsung, yang tidak dapat disebutkan di sini atas bantuan dan perhatiannya selama penyusunan Tugas Akhir ini.
    \item tambahkan sesuai kebutuhan
    % \item tambahkan sesuai kebutuhan
\end{enumerate}

Akhir kata, dengan keterbatasan yang ada pada penulis tentunya masih banyak kekurangan dan masih jauh dari kesempurnaan, hanya Allah SWT yang memiliki segala kesempurnaan. Oleh sebab itu masukan berupa kritik dan saran yang membangun akan sangat membantu bagi penulis. Semoga skripsi ini dapat memberikan manfaat bagi khasanah pengetahuan Teknologi Informasi di Indonesia.
%% Edit bagian ini sesuai kebutuhan

%% DILARANG EDIT BAGIAN INI
% \begin{flushright}
%     {\kota}, \tglpengesahan\\[1.25cm]
%     \penulis \\
%     \nim
% \end{flushright}
%% DILARANG EDIT BAGIAN INI