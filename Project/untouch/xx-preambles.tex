%==================================================================
% Konfigurasi utama untuk dokumen LaTeX
%==================================================================

%% MENGEDIT BAGIAN INI DAPAT MENGUBAH FORMAT DOKUMEN. HARAP DIHINDARI ATAU LAKUKAN DENGAN HATI-HATI

% Mengatur bahasa dokumen ke Bahasa Indonesia dan encoding karakter
\usepackage[indonesian]{babel}
\usepackage[utf8]{inputenc}
% Pengaturan jarak antar baris dan penyesuaian kotak teks agar rata kiri
\usepackage{setspace}
\usepackage[raggedrightboxes]{ragged2e}
% Paket untuk memasukkan gambar dalam dokumen
\usepackage{graphicx}
\usepackage{float}       % Mengatur posisi gambar dalam teks
% Mengatur indentasi paragraf
\usepackage{indentfirst}  % Memastikan paragraf pertama di setiap section memiliki indentasi
% Paket untuk mengatur indentasi section dan subsection
\usepackage{changepage}
% Mengatur font utama dokumen menjadi Times New Roman
\usepackage{pslatex}
% Pengaturan daftar isi, daftar gambar, dan daftar tabel
\usepackage{chngcntr, tocloft}
% Pengaturan format dan penomoran judul bab
\usepackage{titlesec}
% Pengaturan untuk caption gambar dan tabel
\usepackage[font=small, format=plain, up, textfont=up, tablename=TABEL]{caption}
% \usepackage{caption}
\usepackage{subcaption}
% Pengaturan tabel, multirow, dan ukuran kolom otomatis
\usepackage{tabularx}
\usepackage{multirow}
\usepackage{array}
% Pengaturan margin halaman
\usepackage{geometry}
% Paket untuk notasi matematika
\usepackage{amsmath}
% Untuk halaman berorientasi landscape
\usepackage{rotating}
\usepackage{everypage}
\usepackage{pdflscape}
\usepackage{afterpage}
% Pengaturan untuk list (daftar item dan angka)
\usepackage{enumitem}
% Paket untuk bibliografi menggunakan BibTeX
\usepackage[sort]{natbib}
% Paket untuk tabel yang panjang dan melampaui satu halaman
\usepackage{longtable}
% Paket untuk memasukkan hyperlink dalam dokumen
\usepackage{hyperref}
% Pengaturan label otomatis untuk berbagai elemen (gambar, tabel, persamaan)
\usepackage{cleveref}
% Paket untuk menampilkan kode program
\usepackage{listings}
\usepackage{xcolor}
%paket untuk gambar dengan tikz
\usepackage{tikz}
\usepackage{pgfplots}
\usepackage{pgf-pie}
%untuk if-then
\usepackage{ifthen}
%untuk mesanitasi (sensor) teks
\usepackage{censor}

\usepackage{booktabs}
% Paket untuk menghitung halaman dan referensi
\usepackage{totcount}
\usepackage{lastpage}
\usepackage{refcount}

\graphicspath{{gambar/}} % Menentukan folder default untuk gambar

\setlength\parindent{1cm} % Mengatur jarak indentasi paragraf menjadi 1cm (1 tab)

% Mengatur format penomoran section dan subsection
\renewcommand{\thesection}{\thechapter.\arabic{section}}        % Subsection: 1, 2, 3, ...
\renewcommand{\thesubsection}{\thesection.\arabic{subsection}}  % Subsection: 1.1, 2.1, 3.1, ...

% Pengaturan daftar isi, daftar gambar, dan daftar tabel
\cftsetpnumwidth{1.5em}
\cftsetrmarg{2em}
\setlength{\cftsecnumwidth}{1.5em}
\setlength{\cftsubsecnumwidth}{1.5em}
\renewcommand{\cftchapdotsep}{\cftdotsep}
\setlength{\cftbeforechapskip}{3pt}

% Penyesuaian judul bab dalam daftar isi agar ditampilkan sebagai "BAB"
\renewcommand\cftchappresnum{BAB }
\renewcommand\cftchapaftersnum{}
\newlength\mylen
\settowidth\mylen{\bfseries BAB 1 :\ } % Menyesuaikan lebar penomoran bab
\cftsetindents{chap}{0pt}{\mylen}

% % Pengaturan format dan penomoran judul bab
\titleformat{\chapter}{\doublespacing\fontsize{16pt}{18pt}\bfseries}{\MakeUppercase{\chaptertitlename\ \Roman{chapter}}\filcenter}{0.15cm}{\centering\uppercase}

% % Pengaturan format section dengan adjustwidth otomatis
\titleformat{\section}{\fontsize{14}{16}\bfseries}{\thesection}{0.4cm}{}
\titleformat{\subsection}{\fontsize{14}{16}\bfseries}{\thesubsection}{0.675cm}{}

% Mengatur jarak dan format spacing untuk chapter, section dan subsection
\titlespacing*{\chapter}{0pt}{-1cm}{20pt}
\titlespacing*{\section}{0pt}{10pt}{0cm}
\titlespacing*{\subsection}{2.4em}{10pt}{0cm}

%==================================================================
% CUSTOM ENVIRONMENTS UNTUK INDENTASI SECTION DAN SUBSECTION
%==================================================================

% Environment untuk konten section (indentasi 2.4em)
\newenvironment{sectioncontent}{%
    \begin{adjustwidth}{2.4em}{0pt}%
}{%
    \end{adjustwidth}%
}

% Environment khusus untuk sistematika penulisan
\newenvironment{sistematika}{%
    \begin{sectioncontent}%
    \setlength{\parskip}{0\baselineskip}% Jarak antar paragraf
}{%
    \end{sectioncontent}%
}

% Command untuk membuat item sistematika
% Parameter: {nomor bab romawi}{judul bab}{deskripsi}
\newcommand{\babsistematika}[3]{%
    \noindent\textbf{\makebox[2cm][l]{BAB #1}#2}%\\ Aktifkan bagian ini apabila ingin membuat deskripsi menjadi dalam bentuk paragraf
    % Aktifkan bagian ini apabila ingin membuat deskripsi menjadi dalam bentuk paragraf
    % \hspace{0cm}\noindent\makebox[2cm][l]{}#3\par\vspace{0cm}

    % Aktifkan bagian ini apabila ingin membuat deskripsi tidak dalam bentuk paragraf
    {\leftskip=2cm\noindent#3\par\vspace{0cm}}
}

% Environment untuk konten subsection (indentasi 6.0em)
\newenvironment{subsectioncontent}{%
    \begin{adjustwidth}{6.4em}{0pt}%
}{%
    \end{adjustwidth}%
}

% Environment untuk konten subsubsection (jika diperlukan, indentasi 8.4em)
\newenvironment{subsubsectioncontent}{%
    \begin{adjustwidth}{8.4em}{0pt}%
}{%
    \end{adjustwidth}%
}

% Environment alternatif dengan parameter fleksibel untuk indentasi custom
\newenvironment{customindent}[1]{%
    \begin{adjustwidth}{#1}{0pt}%
}{%
    \end{adjustwidth}%
}

% Macro untuk membuat section dengan konten yang sudah ter-indent otomatis
\newcommand{\indentsection}[2]{%
    \section{#1}%
    \begin{sectioncontent}%
        #2%
    \end{sectioncontent}%
}

% Macro untuk membuat subsection dengan konten yang sudah ter-indent otomatis
\newcommand{\indentsubsection}[2]{%
    \subsection{#1}%
    \begin{subsectioncontent}%
        #2%
    \end{subsectioncontent}%
}

%==================================================================
% PENGATURAN LAINNYA (TETAP SAMA)
%==================================================================

% Menghapus tanda titik dua pada caption
% \captionsetup[figure]{labelsep=space}
% \captionsetup[table]{labelsep=space}
\captionsetup[figure]{
  labelsep=newline,
  singlelinecheck=false,
  justification=centering,
}
\captionsetup[table]{
  labelsep=newline,
  singlelinecheck=false,
  justification=centering,
}

% Mengatur nomor caption gambar dan tabel sesuai bab
\renewcommand{\thefigure}{\arabic{chapter}.\arabic{figure}}
\renewcommand{\thetable}{\arabic{chapter}.\arabic{table}}

% Mengatur hyphenation (pemisahan kata) agar lebih rapi
\tolerance=1
\emergencystretch=\maxdimen
\hyphenpenalty=10000
\hbadness=10000

% Pengaturan margin halaman
\geometry{
    left=4cm,          % Margin kiri
    top=4cm,           % Margin atas
    right=3cm,         % Margin kanan
    bottom=3cm,        % Margin bawah
}

% Pengaturan nomor pada persamaan matematika sesuai bab
%\renewcommand{\theequation}{\arabic{chapter}.\arabic{equation}}
\makeatletter
\renewcommand{\theequation}{\arabic{chapter}.\arabic{equation}}
\renewcommand{\@eqnnum}{\theequation}
\def\tagform@#1{\maketag@@@{#1}} % This line removes the parentheses
\makeatother

% Pengaturan untuk list (daftar item dan angka)
\setlist{nosep} % Menghilangkan jarak antar item dalam list
\newenvironment{packed_enum}{ % Membuat lingkungan untuk daftar bernomor
    \begin{enumerate}[leftmargin=1.5\parindent]
        \setlength{\itemsep}{0pt}
        \setlength{\parskip}{0pt}
        \setlength{\parsep}{0pt}
        }{\end{enumerate}}

\newenvironment{packed_item}{ % Membuat lingkungan untuk daftar berpoin
    \begin{itemize}[leftmargin=1.375\parindent]
        \setlength{\itemsep}{0pt}
        \setlength{\parskip}{0pt}
        \setlength{\parsep}{0pt}
        }{\end{itemize}}

% Paket untuk bibliografi menggunakan BibTeX
\bibliographystyle{newapa} % Ganti dengan apalike jika menggunakan Overleaf

% Pengaturan label otomatis untuk berbagai elemen (gambar, tabel, persamaan)
\crefname{figure}{gambar}{gambar}
\Crefname{figure}{Gambar}{Gambar}
\crefname{table}{tabel}{tabel}
\Crefname{table}{Tabel}{Tabel}
\crefformat{equation}{persamaan~#2#1#3}
\crefname{equation}{persamaan}{persamaan}
\Crefname{equation}{Persamaan}{Persamaan}
\crefname{lstlisting}{kode}{kode}
\Crefname{lstlisting}{Kode}{Kode}

%\AtBeginDocument{\renewcommand{\lstlistingname}{Kode}} 
%\AtBeginDocument{\renewcommand{\thelstlisting}{\thechapter.\arabic{lstlisting}}}
%\AtBeginDocument{\renewcommand{\thelstlisting}{\arabic{chapter}.\arabic{lstlisting}.}}
\AtBeginDocument{\renewcommand{\lstlistingname}{Kode}}
\AtBeginDocument{\renewcommand{\thelstlisting}{\arabic{chapter}.\arabic{lstlisting}}}

\captionsetup[lstlisting]{
  format=plain,
  labelfont=bf,
  justification=centering,
  singlelinecheck=false,
  labelsep=space
}

% Paket untuk menampilkan kode program
\definecolor{codegreen}{rgb}{0,0.6,0}
\definecolor{codegray}{rgb}{0.5,0.5,0.5}
\definecolor{codepurple}{rgb}{0.58,0,0.82}
\definecolor{backcolour}{rgb}{0.95,0.95,0.92}
\lstdefinestyle{mystyle}{
    backgroundcolor=\color{backcolour},   
    commentstyle=\color{codegreen},
    keywordstyle=\color{magenta},
    numberstyle=\tiny\color{codegray},
    stringstyle=\color{codepurple},
    basicstyle=\ttfamily\footnotesize,
    columns=fullflexible,
    breakatwhitespace=false,         
    breaklines=true,                 
    captionpos=b,                    
    keepspaces=true,                 
    numbers=left,                    
    numbersep=5pt,                  
    showspaces=false,                
    showstringspaces=false,
    showtabs=false,                  
    tabsize=2,
    lineskip=-1pt
}
\lstset{style=mystyle}

%paket untuk gambar dengan tikz
\pgfplotsset{compat=1.18}
\usetikzlibrary{shapes.geometric, arrows}
\tikzstyle{startstop} = [rectangle, rounded corners, minimum width=3cm, minimum height=1cm,text centered, draw=black, fill=red!30]
\tikzstyle{process} = [rectangle, minimum width=3cm, minimum height=1cm, text centered, draw=black, fill=orange!30]
\tikzstyle{decision} = [diamond, minimum width=3cm, minimum height=1cm, text centered, draw=black, fill=green!30]
\tikzstyle{arrow} = [thick,->,>=stealth]

% Mendaftarkan counter untuk berbagai elemen
\regtotcounter{table}       % Counter untuk tabel
\regtotcounter{figure}      % Counter untuk gambar

% Counter custom untuk lampiran dan pustaka
\newcounter{lampiran}
\regtotcounter{lampiran}

\newcounter{pustaka}
\regtotcounter{pustaka}

% Command untuk mendapatkan halaman roman terakhir
\makeatletter
\newcommand{\savelastfrontpage}{%
    \immediate\write\@auxout{\noexpand\newlabel{LastFrontPage}{{}{\arabic{page}}}}%
}

% Command untuk mendapatkan jumlah halaman roman (dalam angka, dikurangi 1)
\newcommand{\getromanpagecount}{%
    \ifcsname r@LastFrontPage\endcsname
        \the\numexpr\getpagerefnumber{LastFrontPage}-1\relax%
    \else
        0% Default jika belum ada referensi
    \fi
}

% Command untuk mendapatkan halaman roman terakhir dalam format roman (dikurangi 1)
\newcommand{\getromanpagelast}{%
    \ifcsname r@LastFrontPage\endcsname
        \expandafter\@roman\expandafter{\the\numexpr\getpagerefnumber{LastFrontPage}-1\relax}%
    \else
        0% Default jika belum ada referensi
    \fi
}
\makeatother

% UNTUK NATBIB (yang Anda gunakan) - Menghitung pustaka otomatis
\newwrite\tempfile
\def\bibcounter{0}

% Hook untuk menghitung referensi otomatis
\let\oldbibitem\bibitem
\renewcommand{\bibitem}[2][\empty]{%
    \stepcounter{pustaka}%
    \ifx#1\empty
        \oldbibitem{#2}%
    \else
        \oldbibitem[#1]{#2}%
    \fi
}

% Perintah untuk menambah counter lampiran (panggil di setiap lampiran)
\newcommand{\addlampiran}{\stepcounter{lampiran}}

% Definisi untuk positioning absolut nomor halaman
\def\PageTopMargin{1in}
\def\PageLeftMargin{1in}
\newcommand\atxy[3]{%
 \AddThispageHook{\smash{\hspace*{\dimexpr-\PageLeftMargin-\hoffset+#1\relax}%
  \raisebox{\dimexpr\PageTopMargin+\voffset-#2\relax}{#3}}}}

%% MENGEDIT BAGIAN INI DAPAT MENGUBAH FORMAT DOKUMEN. HARAP DIHINDARI ATAU LAKUKAN DENGAN HATI-HATI