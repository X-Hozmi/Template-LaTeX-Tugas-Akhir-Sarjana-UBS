%==================================================================
% Ini adalah bab 1
% Silahkan edit sesuai kebutuhan, baik menambah atau mengurangi \section, \subsection
%==================================================================

\chapter[PENDAHULUAN]{\\ PENDAHULUAN}

Bab ini merupakan penjelasan secara umum, ringkas, dan padat yang menggambarkan dengan tepat isi usulan penelitian yang meliputi:

\section{Latar Belakang Masalah}
\begin{sectioncontent}
    \hspace{\parindent}Latar belakang masalah berisi uraian mengenai alasan memilih topik skripsi tersebut, hal yang menjadi perhatian dan harapan peneliti dari hasil penelitian yang akan dilakukan. Isi latar belakang masalah biasanya mempunyai urutan sebagai berikut:
    
    \begin{enumerate}[leftmargin=0.5cm,label=\alph*.]
        \item Pernyataan tentang gejala/fenomena yang akan diteliti, boleh diangkat dari masalah teoritis atau dari masalah praktis.
        \item Penjelasan tentang alasan pemilihan topik tersebut, atau situasi yang melatarbelakangi munculnya permasalahan yang akan dicarikan solusinya.
        \item Penjelasan bahwa penelitian yang dilakukan memang belum pernah dilakukan atau jika sudah ada penelitian semacam itu perlu dijelaskan perbedaan nyata dengan penelitian sebelumnya atau penjelasan pemilihan metodologi yang dipilih dalam melaksanakan penelitian tersebut.
        \item Penjelasan tentang tujuan dan manfaat yang akan diperoleh setelah penelitian berhasil dilakukan.
    \end{enumerate}
\end{sectioncontent}

\section{Identifikasi Masalah dan Pembatasan Masalah}

\subsection{Identifikasi Masalah}
\begin{subsectioncontent}
    \hspace{\parindent}Kegiatan mengenali sejumlah masalah yang dapat dicarikan jawabannya melalui penelitian. Mengenali masalah ini tertumpu pada masalah pokok yang tercermin pada bagian latar belakang masalah di atas.
\end{subsectioncontent}

\subsection{Pembatasan Masalah}
\begin{subsectioncontent}
    \hspace{\parindent}Bagian ini terkait dengan Identifikasi Masalah diatas. Dengan keterbatasan yang ada pada peneliti maka semua masalah yang telah diidentifikasi tidak dapat diteliti semua, melainkan hanya terbatas pada beberapa masalah saja.
\end{subsectioncontent}

\subsection{Perumusan Masalah}
\begin{subsectioncontent}
    \hspace{\parindent}Rumusan masalah merupakan inti masalah yang menjadi materi pokok penelitian dalam bentuk narasi, inti masalah dapat dinyatakan sebagaimana yang telah disampaikan dalam identifikasi dan batasan masalah, namun telah dilengkapi dengan pernyataan lain sebagaimana yang dikemukakan dalam  batasan masalah.
\end{subsectioncontent}

\section{Tujuan dan Manfaat Penelitian}

\subsection{Tujuan}
\begin{subsectioncontent}
    \hspace{\parindent}Merupakan suatu penjelasan tentang tujuan yang akan dilaksanakan terkait dengan pengembangan keilmuan praktis serta kebijakan dari masalah yang akan diteliti. Tujuan penelitian berisi penjelasan tentang tujuan yang "spesifik" atau target yang ingin dicapai. Pengertian "spesifik" diimplementasikan dengan memakai ungkapan-ungkapan yang jelas, akurat, dan tidak menimbulkan kesalahan interpretasi.
\end{subsectioncontent}

\subsection{Manfaat}
\begin{subsectioncontent}
    \hspace{\parindent}Merupakan suatu penjelasan tentang manfaat penelitian yang akan dilaksanakan terkait dengan pengembangan keilmuan atau manfaat praktis serta kebijakan dari masalah yang akan diteliti. Manfaat penelitian berisi penjelasan tentang manfaat yang akan didapat oleh pihak yang baik terkait langsung ataupun tidak.
\end{subsectioncontent}

\section{Sistematika Penulisan}
\begin{sistematika}
    \babsistematika{I}{PENDAHULUAN}{Berisi latar belakang masalah, identifikasi dan pembatasan masalah, perumusan masalah, tujuan dan manfaat penelitian, serta sistematika penulisan.}
    
    \babsistematika{II}{LANDASAN TEORI}{Menjelaskan tentang landasan teori yang digunakan dalam penelitian dan kerangka pemikiran.}
    
    \babsistematika{III}{METODE PENELITIAN}{Menjelaskan metode penelitian yang digunakan, mulai dari pengumpulan data, analisis kebutuhan, perancangan sistem, hingga implementasi dan pengujian.}
    
    \babsistematika{IV}{HASIL DAN PEMBAHASAN}{Menyajikan hasil penelitian, serta pembahasan mengenai efektivitas dan manfaat penelitian yang dilakukan.}
    
    \babsistematika{V}{KESIMPULAN DAN SARAN}{Berisi kesimpulan dari hasil penelitian serta saran untuk pengembangan lebih lanjut.}
\end{sistematika}