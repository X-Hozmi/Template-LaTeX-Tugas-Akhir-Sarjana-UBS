%==================================================================
% Ini adalah bab 3
% Silahkan edit sesuai kebutuhan, baik menambah atau mengurangi \section, \subsection
%==================================================================

\chapter[METODE PENELITIAN]{\\ METODE PENELITIAN}

Bab ini merupakan penjelasan tentang karakteristik utama dari penelitian yang berupa penyampaian jenis penelitian yang berupa penelitian eksploratif, eksplainatif, deskriptif kualitatif,  dan deskriptif kuantitatif.

Bab ini juga merupakan penjabaran lebih rinci tentang metode penelitian yang secara garis besar telah disinggung pada bab pendahuluan. Pembatasan istilah yang ada pada judul dan variabel yang dilibatkan dalam penelitian juga dijelaskan dalam bab ini. Semua prosedur, proses, dan hasil penelitian, sejak persiapan hingga penelitian berakhir, merupakan isi bab ini. Termasuk dalam bab ini adalah laporan mengenai instrumen yang digunakan beserta variabel dan reabilitasnya. Sangat penting untuk disajikan disini adalah pola alasan dengan disertai pembuktiannya jika mungkin, mengapa sesuatu teknik atau prosedur/metode dipilih oleh penulis sehingga menyakinkan para pembaca bahwa pilihan tersebut memang merupakan teknik atau prosedur yang paling tepat pada saat itu.

\section{Analisa Kebutuhan}
\begin{sectioncontent}
    \hspace{\parindent}Merupakan suatu penjelasan tentang apa saja kebutuhan pengguna dalam mengimplementasi suatu sistem yang berisi suatu uraian lengkap tentang business knowledge dan business function. Analisa kebutuhan dapat berupa metode formulasi / rumus yang akan digunakan dalam penelitian dan atau Pengumpulan Data diuraikan tentang metode pengumpulan data, baik data primer maupun sekunder ( pengamatan atau observasi, angket atau kuesioner, wawancara atau interview, dokumen atau sumber-sumber yang sudah ada antara lain yang berasal dari website resmi, publikasi pemerintah, lembaga penelitian dsb).
\end{sectioncontent}

\section{Perancangan Penelitian}
\begin{sectioncontent}
    \hspace{\parindent}Merupakan suatu penjelasan tentang metode penelitian yang akan  digunakan untuk Software seperti Rapid Application Development, Waterfall, Extreme Programming, dll. Untuk metode penelitian berbasis Networking seperti : Network Development Life Cycle, Prepare Plan Design Implement Operate and Optimaze, dll.
\end{sectioncontent}

\section{Teknik Analisis}
\begin{sectioncontent}
    \hspace{\parindent}Merupakan suatu penjelasan tentang bagaimana sistem, pengolahan data, rancang bangun, pengujian selain metode yang diterapkan pada perancangan penelitian.
\end{sectioncontent}

\section{Jadwal dan Biaya Penelitian}
\begin{sectioncontent}
    \hspace{\parindent}Merupakan suatu penjelasan tentang jadwal penelitian yang disajikan dalam bentuk matriks, sehingga mudah dan cepat dicermati pembacanya. Jadwal penelitian disampaikan secara ringkas, jelas, dan realistik.  Dalam matriks jadwal penelitian ditunjukkan tahap-tahap penelitian, rincian kegiatan pada setiap tahap, dan waktu yang diperlukan untuk melaksanakan setiap tahap penelitian. Sedangkan biaya penelitian disampaikan sebagai bahan pertimbangan portofolionya.
\end{sectioncontent}